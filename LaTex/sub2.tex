\section{Classification}
\subsection{The problem explained}
In this section, we will solve a classification problem using the \textit{Glass Identification} dataset. Specifically, we aim to classify the nominal attribute \texttt{Type of glass} based on the remaining attributes of the data. This attribute can assume seven distinct values as discussed in \textit{Project 1}, and as such, this is a multi-class classification problem. The reasoning behind the choice of this attribute as the target attribute is quite simple; while all the other attributes are continuous, this is, as mentioned, nominal. Furthermore, in real-world applications, the ability to classify specifically this attribute would in most cases be of most relevance, particularly in terms of crime solving which indeed is where this dataset stems from.  

\subsection{Comparison of different models}
In the following, the performance of 3 distinct models will be evaluated and pairwise compared. Aside from the logistic regression and the classical baseline model, an artificial neural network has been trained as the 